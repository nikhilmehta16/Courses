%Credits for some parts of this macro file goes to [Prof Ramprasad Saptharishi](http://www.tcs.tifr.res.in/~ramprasad/)

\setcounter{secnumdepth}{0}
\newcounter{partCounter}
\newcounter{questionCounter}
\setcounter{questionCounter}{1}
\nobreak\extramarks{Question \arabic{questionCounter}}{}\nobreak{}

%
% Homework Problem Environment
%
% This environment takes an optional argument. When given, it will adjust the
% problem counter. This is useful for when the problems given for your
% assignment aren't sequential.
%
\newenvironment{question}[1][-1]{
    \ifnum#1>0
        \setcounter{questionCounter}{#1}
    \fi
    \section{Question \arabic{questionCounter}}
    \setcounter{partCounter}{1}
    
}
{
    
}

\newenvironment{solution}{
    \subsection{Solution}
}
{
	\stepcounter{questionCounter}    
	\newpage
}

\numberwithin{equation}{questionCounter}
\declaretheoremstyle[bodyfont=\it,qed=\qedsymbol]{noproofstyle} 

\declaretheorem[numberlike=equation]{axiom}

\declaretheorem[numberlike=equation]{observation}
\declaretheorem[numberlike=equation,style=noproofstyle,name=Observation]{observationwp}
\declaretheorem[name=Observation,numbered=no]{observation*}

\declaretheorem[numberlike=equation]{fact}
\declaretheorem[numberlike=equation]{subclaim}

\declaretheorem[numberlike=equation]{theorem}
\declaretheorem[numberlike=equation,style=noproofstyle,name=Theorem]{theoremwp}
\declaretheorem[name=Theorem,numbered=no]{theorem*}

\declaretheorem[numberlike=equation]{lemma}
\declaretheorem[name=Lemma,numbered=no]{lemma*}
\declaretheorem[numberlike=equation,style=noproofstyle,name=Lemma]{lemmawp}

\declaretheorem[numberlike=equation]{corollary}
\declaretheorem[name=Corollary,numbered=no]{corollary*}
\declaretheorem[numberlike=equation,style=noproofstyle,name=Corollary]{corollarywp}

\declaretheorem[numberlike=equation]{proposition}
\declaretheorem[name=Proposition,numbered=no]{proposition*}
\declaretheorem[numberlike=equation,style=noproofstyle,name=Proposition]{propositionwp}

\declaretheorem[numberlike=equation]{claim}
\declaretheorem[name=Claim,numbered=no]{claim*}
\declaretheorem[numberlike=equation,style=noproofstyle,name=Claim]{claimwp}

\declaretheorem[numberlike=equation]{conjecture}
\declaretheorem[name=Conjecture,numbered=no]{conjecture*}

\declaretheoremstyle[
    headfont=\bfseries, 
    notebraces={[}{]},
    bodyfont=\normalfont\itshape,
    headpunct={},
    postheadspace=\newline,
    postheadhook={\textcolor{black}{\rule[.6ex]{\linewidth}{0.4pt}}\\},
    spacebelow=\parsep,
    spaceabove=\parsep,
    mdframed={
        backgroundcolor=black!5, 
            linecolor=black!80, 
            innertopmargin=6pt,
            roundcorner=5pt, 
            innerbottommargin=6pt, 
            skipbelow=\parsep, 
            skipbelow=\parsep } 
]{myopenprobstyle}

\declaretheorem[name=Open Problem,style=myopenprobstyle]{openproblem}
\declaretheorem[name=Exercise,style=exercisestyle]{exercise}
\numberwithin{openproblem}{section}
\numberwithin{exercise}{section}

\declaretheoremstyle[bodyfont=\it,qed=$\lozenge$]{defstyle} 

\declaretheorem[numberlike=equation,style=defstyle]{definition}
\declaretheorem[unnumbered,name=Definition,style=defstyle]{definition*}

\declaretheorem[numberlike=equation,style=defstyle]{example}
\declaretheorem[unnumbered,name=Example,style=defstyle]{example*}

\declaretheorem[numberlike=equation,style=defstyle]{notation}
\declaretheorem[unnumbered,name=Notation=defstyle]{notation*}

\declaretheorem[numberlike=equation,style=defstyle]{construction}
\declaretheorem[unnumbered,name=Construction,style=defstyle]{construction*}

\declaretheorem[numberlike=equation,style=defstyle]{remark}
\declaretheorem[unnumbered,name=Remark,style=defstyle]{remark*}

\newenvironment{myproof}[1]%
{\vspace{1ex}\noindent{\em Proof.}\hspace{0.5em}\def\myproof@name{#1}}%
{\hfill{\tiny \qed\ (\myproof@name)}\vspace{1ex}}
\newenvironment{proof-sketch}{\medskip\noindent{\em Sketch of Proof.}\hspace*{1em}}{\qed\bigskip}
\newenvironment{proof-attempt}{\medskip\noindent{\em Proof attempt.}\hspace*{1em}}{\bigskip}
\newenvironment{proofof}[1]{\medskip\noindent\emph{Proof of #1.}\hspace*{1em}}{\qed\bigskip}

\newcommand{\inparen }[1]{\left(#1\right)}             %\inparen{x+y}  is (x+y)
\newcommand{\inbrace }[1]{\left\{#1\right\}}           %\inbrace{x+y}  is {x+y}
\newcommand{\insquare}[1]{\left[#1\right]}             %\insquar{x+y}  is [x+y]
\newcommand{\inangle }[1]{\left\langle#1\right\rangle} %\inangle{A}    is <A>

\newcommand{\abs}[1]{\left|#1\right|}                  %\abs{x}        is |x|
\newcommand{\norm}[1]{\left\Vert#1\right\Vert}         %\norm{x}       is ||x||

\newcommand{\fspan}{\operatorname{\F\text{-span}}}

\newcommand{\union}{\cup}
\newcommand{\Union}{\bigcup}
\newcommand{\intersection}{\cap}
\newcommand{\Intersection}{\bigcap}

\newcommand{\ceil}[1]{\lceil #1 \rceil}
\newcommand{\floor}[1]{\lfloor #1 \rfloor}


\newcommand{\eqdef}{\stackrel{\text{def}}{=}}
\newcommand{\setdef}[2]{\inbrace{{#1}\ : \ {#2}}}      % E.g: \setdef{x}{f(x) = 0}
\newcommand{\set}[1]{\inbrace{#1}}
\newcommand{\innerproduct}[2]{\left\langle{#1},{#2}\right\rangle} %\innerproduct{x}{y} is <x,y>.
\newcommand{\zo}{\inbrace{0,1}}                        % Well just something that is used often!
\newcommand{\parderiv}[2]{\frac{\partial #1}{\partial #2}}
\newcommand{\pderiv}[2]{\partial_{#2}\inparen{#1}}
\newcommand{\zof}[2]{\inbrace{0,1}^{#1}\longrightarrow \inbrace{0,1}^{#2}}

\newcommand*{\mywedge}{\mathord{\wedge}}


% Commonly used blackboard letters
\newcommand{\FF}{\mathbb{F}}
\newcommand{\F}{\mathbb{F}}
\newcommand{\N}{\mathbb{N}}
\newcommand{\Q}{\mathbb{Q}}
\newcommand{\Z}{\mathbb{Z}}
\newcommand{\R}{\mathbb{R}}
\newcommand{\C}{\mathbb{C}}
\newcommand{\RR}{\mathbb{R}}
\newcommand{\E}{\mathop{\mathbb{E}}}


%% Bold letters
\newcommand{\veca}{\mathbf{a}}
\newcommand{\vecb}{\mathbf{b}}
\newcommand{\vecc}{\mathbf{c}}
\newcommand{\vecd}{\mathbf{d}}
\newcommand{\vece}{\mathbf{e}}
\newcommand{\vecf}{\mathbf{f}}
\newcommand{\vecg}{\mathbf{g}}
\newcommand{\vech}{\mathbf{h}}
\newcommand{\veci}{\mathbf{i}}
\newcommand{\vecj}{\mathbf{j}}
\newcommand{\veck}{\mathbf{k}}
\newcommand{\vecl}{\mathbf{l}}
\newcommand{\vecm}{\mathbf{m}}
\newcommand{\vecn}{\mathbf{n}}
\newcommand{\veco}{\mathbf{o}}
\newcommand{\vecp}{\mathbf{p}}
\newcommand{\vecq}{\mathbf{q}}
\newcommand{\vecr}{\mathbf{r}}
\newcommand{\vecs}{\mathbf{s}}
\newcommand{\vect}{\mathbf{t}}
\newcommand{\vecu}{\mathbf{u}}
\newcommand{\vecv}{\mathbf{v}}
\newcommand{\vecw}{\mathbf{w}}
\newcommand{\vecx}{\mathbf{x}}
\newcommand{\vecy}{\mathbf{y}}
\newcommand{\vecz}{\mathbf{z}}

%% Calligraphic Letters
\newcommand{\cala}{\mathcal{A}}
\newcommand{\calb}{\mathcal{B}}
\newcommand{\calc}{\mathcal{C}}
\newcommand{\cald}{\mathcal{D}}
\newcommand{\cale}{\mathcal{E}}
\newcommand{\calf}{\mathcal{F}}
\newcommand{\calg}{\mathcal{G}}
\newcommand{\calh}{\mathcal{H}}
\newcommand{\cali}{\mathcal{I}}
\newcommand{\calj}{\mathcal{J}}
\newcommand{\calk}{\mathcal{K}}
\newcommand{\call}{\mathcal{L}}
\newcommand{\calm}{\mathcal{M}}
\newcommand{\caln}{\mathcal{N}}
\newcommand{\calo}{\mathcal{O}}
\newcommand{\calp}{\mathcal{P}}
\newcommand{\calq}{\mathcal{Q}}
\newcommand{\calr}{\mathcal{R}}
\newcommand{\cals}{\mathcal{S}}
\newcommand{\calt}{\mathcal{T}}
\newcommand{\calu}{\mathcal{U}}
\newcommand{\calv}{\mathcal{V}}
\newcommand{\calw}{\mathcal{W}}
\newcommand{\calx}{\mathcal{X}}
\newcommand{\caly}{\mathcal{Y}}
\newcommand{\calz}{\mathcal{Z}}

%% Bar characters
\newcommand{\bara}{\overline{a}}
\newcommand{\barb}{\overline{b}}
\newcommand{\barc}{\overline{c}}
\newcommand{\bard}{\overline{d}}

\newcommand{\barx}{\overline{x}}
\newcommand{\bary}{\overline{y}}
\newcommand{\barz}{\overline{z}}

\newcommand{\spaced}[1]{\quad #1 \quad}


%% Footer numbering with brackets
\renewcommand*{\thefootnote}{[\arabic{footnote}]}

%% Add footnote without number reference
\newcommand\blfootnote[1]{%
  \begingroup
  \renewcommand\thefootnote{}\footnote{#1}%
  \addtocounter{footnote}{-1}%
  \endgroup
}
