\documentclass[12pt]{article}
\usepackage{fullpage}
\usepackage{amsthm}
%\usepackage{amsmath}
\usepackage{mathtools}
\usepackage{amssymb}
\usepackage{scrextend}
\usepackage{todonotes}
\usepackage{tikz}
\usepackage{thmtools,thm-restate}
\usepackage{setspace}
\usepackage[colorlinks]{hyperref}
\usepackage{extramarks}
\usepackage{enumerate}
\usepackage[outline]{contour}

\input{./bib/customurlbst/bibmacros}

% Update these details with your Group Number, Member Names (Roll No.).
\newcommand{\GroupNum}{261}
\newcommand{\MembAName}{Kurt G\"odel (280406)}
\newcommand{\MembBName}{Bertrand Russell (180572)}
\newcommand{\MembCName}{Alonzo Church (140603)}

% Font
\usepackage[rm,light]{roboto}
\usepackage[T1]{fontenc}
% Or you could use ...
% \usepackage{mathpazo}

\input{./macros}
\allowdisplaybreaks

\begin{document}
    \noindent
    \newcommand{\courseNumb}{CS201}
\newcommand{\courseName}{Mathematics For Computer Science}
\newcommand{\subDate}{}
\newcommand{\assignNumb}{1}

\begin{minipage}[t]{0.50\linewidth}
    \begin{flushleft}
        {\huge \textbf{\courseNumb}}\\
        {\large \courseName}\\
        {\normalsize Indian Institute of Technology, Kanpur}\\
        \rule{0mm}{8mm}%
        {\large \itshape Group Number: \GroupNum}\\
        {\normalsize \textit{\MembAName, \MembBName, \MembCName}}
    \end{flushleft}

\end{minipage}
\hfill
\begin{minipage}[t]{0.40\linewidth}
    \centering
    {\huge Assignment}\\ \rule{0mm}{15mm} \scalebox{5}{\assignNumb}\\~\\
        Date of Submission: \subDate

\end{minipage}

\rule{0mm}{0.5mm}%

{\centering \rule{0.99\linewidth}{1pt} }
    \onehalfspace

    \begin{question}
        Let $S = \{(a,b,c) | a,b,c \in \Z\}$ be the set of all triplets of integers. Show that $|S| = \aleph_0$.
    \end{question}

    \begin{solution}
        Your solution goes here.
    \end{solution}

    \begin{question}
        For any $a,b,c,d \not \in \{-\infty, \infty\}$, show that $|[a,b]| = |[c,d]|$ where $[x,y]$ is the set of all real numbers between $x$ and $y$
    \end{question}
    \begin{solution}
        Your solution goes here.
    \end{solution}

    \begin{question}
        Show that $|[0,1]| = \aleph_1$ where $[0,1]$ is the set of all real numbers between 0 and 1.
    \end{question}
    
    \begin{solution}
        Your solution goes here.
    \end{solution}

    \begin{question}
        Show that $|\{0,1\}^*| = \aleph_1$ where $\{0,1\}^*$ is the set of all binary strings of infinite length.
    \end{question}
    
    \begin{solution}
        Your solution goes here.
    \end{solution}

    \begin{question}
        Suppose $R$ is a partial order on $A$ and $S$ be a partial order on $B$. Let $L$ be a binary relation on $A \times B$ defined as $(a,b)L(a',b')$ iff
        \begin{itemize}
            \item $a \neq a'$ and $aRa'$
            \item $a = a'$ and $bSb'$.
        \end{itemize}
        Show that $L$ is also a partial order on $A \times B$. Is it a total order?
    \end{question}

    \begin{solution}
        Your solution goes here.
    \end{solution}

    \begin{question}
        Let $R$ be a binary relation on $\N$ defined as $aRb$ if $b = 2^k a$ where $k$ is a non-negative integer. Show that $R$ is a partial order on $\N$.
    \end{question}
    
    \begin{solution}
        Your solution goes here.
    \end{solution}

    \begin{question}
        Let $n$ be a positive integer. Consider the relation $\equiv_n$ on $\Z$ such that $a \equiv_n b \iff a = b \mod \ n$. Show that $\equiv_n$ is an equivalence relation on $\Z$. What are the equivalence classes?
    \end{question}

    \begin{solution}
        Your solution goes here.
    \end{solution}

    \begin{question}
        Consider the relation $S$ on $\N$ such that $aSb \iff ab$ is a perfect square. Show that $S$ is an equivalence relation on $\N$. What are the equivalence classes?
    \end{question}
    
    \begin{solution}
        Your solution goes here.
    \end{solution}

    \begin{question}
        There was an ambiguity in the definition of a well-ordering in the lectures. It is clarified here. 
        
        A {\em well-ordering} $R$ on set $A$ is a partial order such that for every subset $B\subseteq A$, $B$ has an element $m$ such that $mRb$ for every $b\in B$.
        
        In lecture 6, a partial order is shown to be a well-ordering twice: once during proof of the implication that Axiom of Choice implies Zorn's Lemma, and next during proof of the implication that Zorn's Lemma implies Well-Ordering Principle. Redo both these proofs in light of the above clarification.
    \end{question}

    \begin{solution}
        Your solution goes here.
    \end{solution}

    \bibliographystyle{./bib/customurlbst/alphaurlpp}
    \bibliography{./bib/references}
\end{document}  