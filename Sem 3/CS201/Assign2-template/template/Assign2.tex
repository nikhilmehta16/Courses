\documentclass[12pt]{article}
\usepackage{fullpage}
\usepackage{amsthm}
%\usepackage{amsmath}
\usepackage{mathtools}
\usepackage{amssymb}
\usepackage{scrextend}
\usepackage{todonotes}
\usepackage{tikz}
\usepackage{thmtools,thm-restate}
\usepackage{setspace}
\usepackage[colorlinks]{hyperref}
\usepackage{extramarks}
\usepackage{enumerate}
\usepackage[outline]{contour}
\usepackage{fancyhdr}

\input{./bib/customurlbst/bibmacros}

% Update these details with your Group Number, Member Names (Roll No.).
\newcommand{\GroupNum}{261}
\newcommand{\MembAName}{Kurt G\"odel (280406)}
\newcommand{\MembBName}{Bertrand Russell (180572)}
\newcommand{\MembCName}{Alonzo Church (140603)}

% Font
\usepackage[rm,light]{roboto}
\usepackage[T1]{fontenc}
% Or you could use ...
% \usepackage{mathpazo}

\input{./macros}
\allowdisplaybreaks

\begin{document}
    \noindent
    \newcommand{\courseNumb}{CS201}
\newcommand{\courseName}{Mathematics For Computer Science}
\newcommand{\subDate}{}
\newcommand{\assignNumb}{1}

\begin{minipage}[t]{0.50\linewidth}
    \begin{flushleft}
        {\huge \textbf{\courseNumb}}\\
        {\large \courseName}\\
        {\normalsize Indian Institute of Technology, Kanpur}\\
        \rule{0mm}{8mm}%
        {\large \itshape Group Number: \GroupNum}\\
        {\normalsize \textit{\MembAName, \MembBName, \MembCName}}
    \end{flushleft}

\end{minipage}
\hfill
\begin{minipage}[t]{0.40\linewidth}
    \centering
    {\huge Assignment}\\ \rule{0mm}{15mm} \scalebox{5}{\assignNumb}\\~\\
        Date of Submission: \subDate

\end{minipage}

\rule{0mm}{0.5mm}%

{\centering \rule{0.99\linewidth}{1pt} }
    \onehalfspace
    \thispagestyle{plain}

    \begin{question}
        A \textbf{partition} of $n$ objects is a collection of its mutually disjoint subsets, called \textit{blocks}, whose union gives the whole set. Let $S(n; k_1, k_2, \dots, k_n)$ denote the number of all partitions of $n$ objects with $k_i$ $i$-element blocks (i.e., $k_1 + 2k_2 + \dots + nk_n = n$). In other words, 
        \[k_i = \text{the number of $i$-element blocks in a partition}\]
    Show that $S(n; k_1, k_2, \dots, k_n) = \dfrac{n!}{k_1!k_2!\dots k_n! (1!)^{k_1}(2!)^{k_2}\dots (n!)^{k_n}}$.
    \end{question}

    \begin{solution}
        Your solution goes here.
    \end{solution}

    \begin{question}
        Show that for every $k$, the product of any $k$ consecutive natural numbers is divisible by $k!$.
    \end{question}
    \begin{solution}
        Your solution goes here.
    \end{solution}

    \begin{question}
        Show that the number of pairs $(A, B)$ of distinct subsets of $\{1,2, \dots, n\}$ with $A \subset B$ is $3^n - 2^n$.
    \end{question}
    
    \begin{solution}
        Your solution goes here.
    \end{solution}

    \begin{question}
        There is a set of $2n$ people ($n$ males and $n$ females). A good party is a set with the same number of males and females. How many ways are there to build such a good party?
    \end{question}
    
    \begin{solution}
        Your solution goes here.
    \end{solution}

    \begin{question}
        \begin{enumerate}
            \item Show that the number of integer solution to the equation
                \[x_1 + x_2 + \dots + x_n = k\]
            under the condition that $x_i \geq 0$ for all $i$ is $\binom{n+k-1}{k}$.
            
            \item Let $n$ and $k \geq l$ be positive integers. How many different integer solutions are there to the equation $x_1 + x_2 + \dots + x_n = k$ such that $0 \leq x_i < l$ for all $i$.
        \end{enumerate}
    \end{question}

    \begin{solution}
        Your solution goes here.
    \end{solution}

    \bibliographystyle{./bib/customurlbst/alphaurlpp}
    \bibliography{./bib/references}
\end{document}  